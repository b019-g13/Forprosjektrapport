\documentclass[11pt,a4paper]{report} 

\usepackage[utf8]{inputenc} 
\usepackage[norsk]{babel} 
\usepackage{lipsum,paralist}


\begin{document}
\title{
\LARGE
Forprosjektrapport \\
\vspace{2cm}
Utvikling nettsted for Sirkus Media \\
\vspace{2cm}
\Huge
B019-G13
}
\author{
\LARGE 
Bereket G adhanom \\
\LARGE 
Bjørnar Hagen \\
\LARGE 
Line Sharine Hagen
}
\maketitle

\section*{Prosjektgruppen}

\subsection*{Om Bereket}
Gikk på videregåendeskole som IKT-servicemedarbeider og var lærling hos Rana kommune IT-avdeling. Etter jeg fikk fagbrev begynte jeg på Høgskolen i Østfold på dataingeniørstudiet Y-vei.

Er interessert å jobbe med teknologier som HTML, CSS, PHP og JavaScript.

\subsection*{Om Bjørnar}
Tok fagbrev som IKT-servicemedarbeider hos Optimale Systemer AS i Larvik etter 2 år som lærling. Dette etter å ha fullført utdanning på IKT-linja hos Sandefjord videregående skole. Deretter gikk det videre til Høgskolen i Østfold. Startet da på Y-veien for dataingeniørstudiet.

Jobber nå som daglig leder hos Datahjelpen AS, hvor arbeidsoppgavene innebærer utvikling som full-stack og en del DevOps. Liker å jobbe med teknologier som HTML, CSS, JavaScript, Node, React, PHP og Linux.

\subsection*{Om Line}
Gikk IKT-servicefag på Sandefjord videregående skole. Var etter dette lærling hos Vestfold Fylkeskommune der jeg tok fagbrev og dermed ble utdannet som IKT-servicemedarbeider. Startet deretter på Y-veien på dataingeniørstudiet ved Høgskolen i Østfold. 

Jobber nå som front-end utvikler hos Datahjelpen AS og liker best å jobbe med teknologier som HTML, CSS og JavaScript.

\section*{Oppdragsgiver}

Sirkus Media ble stiftet i 2010 av Hans-Christian Hymer og er et teknologi-, data- og analyseselskap som leverer innovative løsninger for digital markedsføring. Deres kunder er små og store bedrifter i hele Norge som ønsker å få økt salg og lønnsomhet. 

Ved hjelp av deres egenutviklede løsning jobber de med å skaffe konkrete kundehenvendelser til sine kunder. Løsningen når direkte ut til kjøpsklare kunder. Dette innebærer at de utelukkende markedsfører mot personer som, basert på nettadferd, vet at med stor sannsynlighet er på utkikk etter produkter eller tjenester som kundene tilbyr.

Sirkus Media har i dag 2 ansatte og hadde i 2017 en omsetning på 6.4 millioner. 

\section*{Oppdraget}

Beskriv selve oppdraget, hvorfor dette er interessant/viktig for oppdragsgiver, evt. hva slags verktøy dere skal bruke etc. Det er viktig at selve rammen rundt prosjektet blir tydelig. Enkelte oppdragsgivere gir ganske frie tøyler, mens andre har ganske konkrete krav og ønsker.

---

Sirkus Media ønsker å forbedre deres profil på nett slik at potensielle kunder får en bedre forståelse av deres produkter og tjenester og tar kontakt gjennom eget nettsted. Dette er viktig for oppdragsgiver ettersom dagens nettsted har store mangler og ikke inneholder nok informasjon om hva Sirkus Media tilbyr. Sirkus Media har derfor et stort behov for utvikling av en ny nettside. 

Nettstedet skal inkludere en løsning for skjemahenvendelser, beskrive Sirkus Media sitt produkt, hva de tilbyr, resultater de kan vise til, omtaler fra eksisterende kunder.
I tillegg skal nettstedet inneholde informasjon angående deres prosess, live chat og kart integrert med Google Maps.

\subsection*{Verktøy}

Oppdragsgiver har gitt oss helt frie tøyler. Prosjektgruppen har derfor selv valgt å bruke følgende verktøy:

\begin{itemize}
\item HTML
\item CSS
\item SCSS
\item JS
\item PHP
\item Laravel
\item Composer
\item NodeJS
\item BrowserSync
\item ReactJS
\item Git
\item Linux
\item Nginx
\item MariaDB
\item buddy.works
\item letsencrypt
\item Photoshop
\item Amazon Web Services (AWS) s3 
\end{itemize}

\begin{itemize}
\item Alle: VCS: Git
\item Back-end: Laravel. Rest-API
\item Front-end: HTML, CSS/SASS, JS/ReactJS/NodeJS, BrowserSync
\item DevOps: Linux server (hos Heroku først, så TerraHost senere): Ubuntu 18.04 LTS. med Nginx og MariaDB. PHP 7.2.
\item DevOps: CI/CD: https://buddy.works
\item DevOps: SSL/TLS: Lets Encrypt
\item DevOps: Backup: https://ottomatik.io/ og AWS s3
\item DevOps: Lagring: AWS s3
\end{itemize}

\textbf{Laravel} er et PHP web-rammeverk som er basert på Symfony. Det følger Model-view-controller (MVC) designmønsteret. Dette skal vi bruke til å utvikle et REST-API og vil fungere som back-end til vårt nettsted.
\textbf{React JS} er et JavaScript bibliotek for å bygge brukergrensesnitt.

---






---

I denne delen skal dere også ta for dere  tre sentrale aspekter: Formål, leveranser og metode. {\em Formålet} (ofte bare kalt {\em målet}, skal beskrive virkningen av prosjektet på et overordnet plan (f.eks. øke omsetningen i et firma). 
{\em Leveransene} er konkrete resultater (tangibles) som blir produsert underveis (f.eks. programvare med tilhørende brukerdokumentasjon), mao.\ {\em hva} som skal produseres. 
{\em Metoden} er {\em hvordan} formål og leveranser skal oppnås (f.eks. analysere dagens situasjon og designe og utvikle en ny nettbutikk). 
Jo mer teoretisk og ``akademisk'' prosjektet er, jo større vekt må man legge på metoden. Tradisjonelt er det metodiske aspektet relativt nedtonet i et bachelorprosjekt i forhold til et master- eller PhD-prosjekt.
Erfaringsmessig oppfatter studentene dette som en litt fremmed måte å betrakte et prosjekt på, men den er utbredt i både akademia og næringslivet, og gjør det lettere å holde tunga rett i munnen underveis. 

Formålet uttrykkes gjerne som ett hovedmål, og et par-tre delmål som utdyper hovedmålet. Beskrivelsen av et mål starter nesten alltid med et verb.


\subsection*{Formål}

\begin{compactitem}
\item [{\bf Hovedmål}] Forbedre deres profil på nett slik at potensielle kunder får en bedre forståelse av deres produkter og tjenester og tar kontakt gjennom eget nettsted. På et overordnet plan vil dette bidra til å øke omsetningen til oppdragsgiver.
\begin{compactitem}
\item [{\bf  Delmål 1} ] Nam dui ligula, fringilla a, euismod sodales, sollicitudin vel, wisi. Morbi
auctor lorem non justo.
\item [{\bf  Delmål 2} ] BLA Nam dui ligula, fringilla a, euismod sodales, sollicitudin vel, wisi. Morbi
auctor lorem non justo.
\end{compactitem}
\end{compactitem}

\subsection*{Leveranser}

Maecenas lacinia. Nam ipsum ligula, eleifend at, accumsan nec, susci- pit a, ipsum. Morbi blandit ligula feugiat magna. Nunc eleifend consequat lorem. Sed lacinia nulla vitae enim. Pellentesque tincidunt purus vel magna. Integer non enim. Praesent euismod nunc eu purus. Donec bibendum quam in tellus. Nullam cursus pulvinar lectus. Donec et mi. Nam vulputate metus eu enim. Vestibulum pellentesque felis eu massa.

Analyse av gammelt nettsted
Mockup
Database og modeller
Nettsted
Brukerveiledning

\subsection*{Metode}
Maecenas lacinia. Nam ipsum ligula, eleifend at, accumsan nec, susci- pit a, ipsum. Morbi blandit ligula feugiat magna. Nunc eleifend consequat lorem. Sed lacinia nulla vitae enim. Pellentesque tincidunt purus vel magna. Integer non enim. Praesent euismod nunc eu purus. Donec bibendum quam in tellus. Nullam cursus pulvinar lectus. Donec et mi. Nam vulputate metus eu enim. Vestibulum pellentesque felis eu massa.


\section*{Prosjektplan}

Prosjektplanen består av et antall veldefinerte aktiviteter. En aktivitet (task), bør ha disse elementene: 

\begin{compactitem}
\item Tittel/nummer og navn
\item Varighet (startdato, sluttdato)
\item Bemanning 
\item Leveranse(r)
\item Forklarende tekst
\end{compactitem}

Prosjektplanen kan f.eks. presenteres på denne måten\footnote{Det kan jo være greit å lage et Gantt-diagram også.}

\begin{compactdesc}
\item [Aktivitetet 1:] Navnet på aktiviteten
	\begin{compactitem}
	\item Start: XX/XX
	\item Slutt: XX/XX
	\item Bemanning: NN1, NN2, etc
	\item Leveranse: Nullamalesuadaporttitordiam.Donecfeliserat,congue
non, volutpat at, tincidunt tristique, libero. 
	\item Beskrivelse: Nullamalesuadaporttitordiam.Donecfeliserat,congue
non, volutpat at, tincidunt tristique, libero. Vivamus viverra fer- mentum felis. Donec nonummy pellentesque ante. 
	\end{compactitem}
	\item [Aktivitetet X:.] Navnet på aktiviteten
	\begin{compactitem}
	\item Start: XX/XX
	\item Slutt: XX/XX
	\item Bemanning \dots
	\item Leveranse: \dots
	\item Beskrivelse: \dots
	\end{compactitem}

\end{compactdesc}


Det kan være lurt å lage en prioritert plan, dvs. at noen av oppgavene vil bli bli gjennomført hvis det blir tid og anledning til det. Det er vanskelig å planlegge et prosjekt, nettopp derfor kan det være lurt at planen sier at dette vil vi oppnå som et minimum, og så kommer et antall prioriterte oppgaver. Dere vil også antagelig få behov for å re-planlegge underveis.  Det er også fornuftig å gjøre en kort risikoanalyse av prosjekt, og peke på kritiske faktorer og eventuelle flaskehalser. Vær spesielt oppmerksomme på at det er betenkelig å gjøre prosjektet avhenging av utstyr, programvare eller liknende som ikke er tilgjengelig ved prosjektstart.

\section*{Gjennomføring}

Her skal dere  fokusere på hvordan gruppa har tenkt seg gjennomføringen av prosjekt, ut over det som framgår av prosjektplanen. Dette kalles ofte for prosessen. Forholdet til arbeidsgiver står sentralt, hvordan skal dette foregå, hvor ofte skal man møtes, hva slags type tilbakemeldinger ønsker man seg etc.

Det er også viktig å redegjøre for hvordan man tenker seg rollene i selve gruppa, skal det f.eks. være en prosjektleder, skal det være en fast prosjektleder, eller skal rollene rullere? Tenker man å bruke en spesiell metodikk, for eksempel SCRUM, ved gjennomføringen av prosjektet og utviklingen av eventuell kode? Hva slags type infrastruktur er man blitt enige om, hvordan håndteres versjonskontroll og backup, hva med å jobbe distribuert?

En annen klassisk utfordring i et prosjektarbeid er hvordan man takler unntak. Hva gjør man hvis et gruppemedlem blir sykt i en kritisk periode, eller ikke leverer som avtalt? Tenker man seg muligheten av re-planlegging i visse situasjoner?

Til slutt kan det være greit å kommentere kort hvordan man tenker seg forholdet mellom veilederen og gruppa. 

\end{document}
