\documentclass[11pt,a4paper]{report} 

\usepackage[utf8]{inputenc} 
\usepackage[norsk]{babel} 
\usepackage{lipsum,paralist}
\usepackage{enumitem}


\begin{document}
\title{
\LARGE
Forprosjektrapport \\
\vspace{2cm}
Utvikling nettsted for Sirkus Media \\
\vspace{2cm}
\Huge
B019-G13
}
\author{
\LARGE 
Bereket G adhanom \\
\LARGE 
Bjørnar Hagen \\
\LARGE 
Line Sharina Aamodt Hagen
}
\maketitle

\section*{Prosjektgruppen}

Deltagerne i prosjektgruppen ble raskt kjent på det obligatoriske sommerkurset til Y-veien på dataingeniørstudiet. Siden den gang har vi samarbeidet på alle gruppeoppgaver underveis i studiet. Alle deltagerne har alltid samarbeidet godt, og dette var en viktig faktor når vi skulle danne en gruppe for bacheloren. 

\subsection*{Om Bereket}
Gikk på IKT-servicefag på  videregående og var deretter lærling hos IT-avdelingen til Rana kommune. Etter bestått fagbrev begynte jeg på Høgskolen i Østfold på dataingeniørstudiet Y-vei.

Er interessert å jobbe med teknologier som HTML, CSS, PHP og JavaScript.

\subsection*{Om Bjørnar}
Tok fagbrev som IKT-servicemedarbeider hos Optimale Systemer AS i Larvik etter 2 år som lærling. Dette etter å ha fullført IKT-linja hos Sandefjord videregående skole. Deretter gikk det videre til Høgskolen i Østfold. Startet da på Y-veien for dataingeniørstudiet.

Jobber nå som daglig leder hos Datahjelpen AS, hvor arbeidsoppgavene innebærer utvikling som full-stack og en del DevOps. Liker å jobbe med teknologier som HTML, CSS, JavaScript, Node, React, PHP og Linux.

\subsection*{Om Line}
Gikk IKT-servicefag på Sandefjord videregående skole. Var etter dette lærling hos Vestfold Fylkeskommune der jeg tok fagbrev og dermed ble utdannet IKT-servicemedarbeider. Startet deretter på Y-veien på dataingeniørstudiet ved Høgskolen i Østfold. 

Jobber nå som front-end utvikler hos Datahjelpen AS og liker best å jobbe med teknologier som HTML, CSS og JavaScript.

\section*{Oppdragsgiver}

Sirkus Media ble stiftet i 2010 av Hans-Christian Hymer og er et teknologi-, data- og analyseselskap som leverer innovative løsninger for digital markedsføring. Deres kunder er små og store bedrifter i hele Norge som ønsker å få økt salg og lønnsomhet. 

Ved hjelp av deres egenutviklede løsning jobber de med å skaffe konkrete kundehenvendelser til sine kunder. Løsningen når direkte ut til kjøpsklare kunder. Dette innebærer at de utelukkende markedsfører mot personer som, basert på nettadferd, vet at med stor sannsynlighet er på utkikk etter produkter eller tjenester som kundene tilbyr.

Sirkus Media har i dag 2 ansatte og hadde i 2017 en omsetning på 6.4 millioner.

Vår kontaktperson hos oppdragsgiver er Hans-Christian Hymer som både er grunnlegger og daglig leder av Sirkus Media. 

\section*{Oppdraget}

Beskriv selve oppdraget, hvorfor dette er interessant/viktig for oppdragsgiver, evt. hva slags verktøy dere skal bruke etc. Det er viktig at selve rammen rundt prosjektet blir tydelig. Enkelte oppdragsgivere gir ganske frie tøyler, mens andre har ganske konkrete krav og ønsker.

---

Bakgrunnen for dette oppdraget er at Sirkus Media ønsker å forbedre deres profil på nett, slik at potensielle kunder får en bedre forståelse av deres produkter og tjenester og tar kontakt gjennom eget nettsted. Dette er viktig for oppdragsgiver ettersom dagens nettsted har store mangler og ikke inneholder nok informasjon om hva Sirkus Media tilbyr. Sirkus Media har derfor et stort behov for utvikling av en ny nettside. 

Nettstedet skal inkludere en løsning for skjemahenvendelser, beskrive Sirkus Media sitt produkt, hva de tilbyr, resultater de kan vise til, omtaler fra eksisterende kunder.
I tillegg skal nettstedet inneholde informasjon angående deres prosess, live chat og kart integrert med Google Maps. Oppdragsgiver ytret ønsket også å ha muligheten til å logge inn på nettstedet og selv kunne oppdatere innhold.

\subsection*{Verktøy}

Oppdragsgiver har ikke satt krav til verken utforming, bruk av verktøy eller grafisk uttrykk. Vi har derfor fått ganske frie tøyler. Prosjektgruppen har følgelig selv valgt ut de forskjellige verktøyene vi mener passer bedriften og deres behov. Følgende verktøy vurderer vi i utgangspunktet å bruke:

\textit{Laravel} er et \textit{PHP} web-rammeverk som er basert på \textit{Symfony}. Det følger \textit{Model-view-controller} (MVC) designmønsteret. Dette skal vi bruke til å utvikle et \textit{REST-API} og vil fungere som \textit{back-end} til vårt nettsted.
\textit{React JS} er et \textit{JavaScript} bibliotek for å bygge brukergrensesnitt. Dette har vi tenkt å bruke som \textit{front-end} for vår løsning sammen med \textit{HTML} og \textit{CSS}. Før vi begynner å lage front-end har vi tenkt til å lage \textit{mockups} med \textit{Figma}.
For versjonskontroll har vi valgt å bruke \textit{Git}.
\textit{Repositories} lagres hos \textit{Github}, som vi også bruker for prosjektstyring. For  \textit{CI/CD} skal vi bruke \textit{Buddy.works}

Når nettstedet begynner å bli ferdig skal vi laste det opp på en server som kjører \textit{LEMP}-stacken (Linux, Nginx, MySQL og PHP). Vi tenker å levere nettstedet over \textit{HTTP2}-protokollen, noe som gjør at vi trenger et \textit{SSL/TLS}-sertifikat for å servere nettsider over \textit{HTTPS}. Dette skaffer vi gjennom \textit{Let's Encrypt}. For å ta backup av databasen skal vi bruke \textit{Ottomatik.io}, og for backup av filer bruker vi \textit{AWS} (Amazon Web Services) sin \textit{S3} tjeneste, som vi også brukes for lagring av media-filer.

Etter nettstedet er lansert skal vi måle trafikk og bruk av nettsidene med \textit{Google Analytics}.

---






---

I denne delen skal dere også ta for dere  tre sentrale aspekter: Formål, leveranser og metode. {\em Formålet} (ofte bare kalt {\em målet}, skal beskrive virkningen av prosjektet på et overordnet plan (f.eks. øke omsetningen i et firma). 
{\em Leveransene} er konkrete resultater (tangibles) som blir produsert underveis (f.eks. programvare med tilhørende brukerdokumentasjon), mao.\ {\em hva} som skal produseres. 
{\em Metoden} er {\em hvordan} formål og leveranser skal oppnås (f.eks. analysere dagens situasjon og designe og utvikle en ny nettbutikk). 
Jo mer teoretisk og ``akademisk'' prosjektet er, jo større vekt må man legge på metoden. Tradisjonelt er det metodiske aspektet relativt nedtonet i et bachelorprosjekt i forhold til et master- eller PhD-prosjekt.
Erfaringsmessig oppfatter studentene dette som en litt fremmed måte å betrakte et prosjekt på, men den er utbredt i både akademia og næringslivet, og gjør det lettere å holde tunga rett i munnen underveis. 

Formålet uttrykkes gjerne som ett hovedmål, og et par-tre delmål som utdyper hovedmålet. Beskrivelsen av et mål starter nesten alltid med et verb.


\subsection*{Formål}

\begin{compactitem}
\item [{\bf Hovedmål}] Forbedre Sirkus Media sin profil på nett, slik at potensielle kunder får en bedre forståelse av deres produkter og tjenester og dermed tar kontakt gjennom deres nettsted. På et overordnet plan vil dette bidra til å øke omsetningen til oppdragsgiver.
\begin{compactitem}
\item [{\bf  Delmål 1} ] Generere mer trafikk, som fører til at flere kunder tar kontakt med bedriften. 
\item [{\bf  Delmål 2} ] Måle trafikken på nettstedet, slik at man ser hva som fungerer og deretter kan tilpasse informasjonen til brukerne som besøker siden.
\end{compactitem}
\end{compactitem}

\subsection*{Leveranser}

Følgende dokumenter må leveres:
\begin{itemize}
\item Prosjekt og gruppekontrakt
\item Gruppenettsted
\item Forprosjektrapport
\item Første versjon av hoveddokument
\item Andre versjon av hoveddokument
\item Hoveddokument med vedlegg
\item Prosjektplakat
\item Presentasjon
\end{itemize}

I tillegg vil disse konkrete resultatene bli opprettet i løpet av dette prosjektet:

\begin{itemize}
\item Analyse av gammelt nettsted
\item Analyse av konkurrenter
\item Analyse av verktøy
\item PROFIL???
\item Mockups av nytt nettsted
\item Database og modeller
\item Nettsted
\item Analyse av nytt nettsted
\item Sitemap
\item Brukerveiledninger
\end{itemize}

\subsection*{Metode}
Hovedmålet er å forbedre Sirkus Media sin nettprofil. Dette løser vi med å analysere dagens nettside og kartlegger hva som er bra og dårlig. Deretter går vi videre til å analysere konkurrentene sine nettsider og ser hvordan deres løsninger ser ut. 

Det første delmålet er å generere mer trafikk til nettsiden. Da må vi å lage et attraktiv nettsted som følger beste praksis for god søkemotoroptimalisering, semantikk og universell utforming. 

For å oppnå delmålet om måling av trafikk på nettstedet må vi benytte oss av verktøy som Google Analytics. 

Mockups oppnår vi ved å hente inspirasjon fra lignende nettsider og deretter lager et forslag på design til nettstedet.

Database og modeller løser vi ved å først tegne og deretter forbedre databasen ved hjelp av penn og papir. 

\section*{Prosjektplan}

Prosjektet består av en rekke aktiviteter med tidsfrister. Deltageren som har ansvar for oppgaven sørger for levering til avtalt tid. Ved eventuell forsinkelse må deltageren informere resten av gruppen om dette. Aktivitetene er satt opp i prioritert rekkefølge.

\smallskip

\newcounter{aktivitetTeller}
\setcounter{aktivitetTeller}{1}

\begin{compactdesc}
    \item [Aktivitet \arabic{aktivitetTeller}:] Første møte med oppdragsgiver
	\begin{compactitem}
	\item Start: 22.11.2018
	\item Slutt: 22.11.2018
	\item Bemanning: Hele gruppen
	\item Beskrivelse: Danner grunnlaget for prosjektet. Overordnet informasjon om oppdraget.
	\end{compactitem}
	\addtocounter{aktivitetTeller}{1}
	\item [Aktivitet \arabic{aktivitetTeller}:] Gruppenettsted
	\begin{compactitem} 
	\item Start: 18.12.2018
	\item Slutt: 08.01.2019
	\item Bemanning: Hele gruppen
	\item Ansvarlig: Bjørnar
	\item Leveranse: Nettsted
	\item Beskrivelse: Hjemmesiden skal inneholde kontaktinformasjon for deltagerne og en beskrivelse av prosjektet. Etterhvert skal alle dokumenter, media, og kildekode gjøres tilgjengelig.
	\addtocounter{aktivitetTeller}{1}
	\end{compactitem}
	\item [Aktivitet \arabic{aktivitetTeller}:] Forprosjektrapport
	\begin{compactitem}
	\item Start: 08.01.2019
	\item Slutt: 18.01.2019
	\item Bemanning: Hele gruppen
	\item Ansvarlig: Line
	\item Leveranse: Rapport
	\item Beskrivelse: Fastsetter formålet for prosjektet, hva gruppen skal levere og hvordan prosjektet skal utføres.
	\addtocounter{aktivitetTeller}{1}
	\end{compactitem}
	\item [Aktivitet \arabic{aktivitetTeller}:] Andre møte med oppdragsgiver
	\begin{compactitem}
	\item Start: 16.01.2019
	\item Slutt: 16.01.2019
	\item Bemanning: Hele gruppen
	\item Beskrivelse: Konkret informasjon om oppdraget og hva nettstedet skal innholde.
	\addtocounter{aktivitetTeller}{1}
	\end{compactitem}
	\item [Aktivitet \arabic{aktivitetTeller}:] Analysere nåværende nettsted
	\begin{compactitem}
	\item Start: 17.01.2019
	\item Slutt: 18.01.2019
	\item Bemanning: Bjørnar
	\item Leveranse: Rapport
	\item Beskrivelse: Dokumentasjon av analyse med forklaring av test-resultatene. Kjører nettstedet gjennom forskjellige verktøy og analyserer resultatene.
	\addtocounter{aktivitetTeller}{1}
	\end{compactitem}
	\item [Aktivitet \arabic{aktivitetTeller}:] Analysere konkurrenter
	\begin{compactitem}
	\item Start: 21.01.2019
	\item Slutt: 22.01.2019
	\item Bemanning: Line
	\item Leveranse: xxx
	\item Beskrivelse: xxx
	\addtocounter{aktivitetTeller}{1}
	\end{compactitem}
	\item [Aktivitet \arabic{aktivitetTeller}:] Analysere passende verktøy
	\begin{compactitem}
	\item Start: 21.01.2019
	\item Slutt: 27.01.2019
	\item Bemanning: Hele gruppen
	\item Ansvarlig: Bereket
	\item Leveranse: xxx
	\item Beskrivelse: xxx
	\addtocounter{aktivitetTeller}{1}
	\end{compactitem}
	\item [Aktivitet \arabic{aktivitetTeller}:] Planlegge innhold
	\begin{compactitem}
	\item Start: 28.01.2019
	\item Slutt: 29.01.2019
	\item Bemanning: Hele gruppen
	\item Ansvarlig: Line
	\item Leveranse: xxx
	\item Beskrivelse: Hvilke sider som skal være, og hva som skal være på de
	\addtocounter{aktivitetTeller}{1}
	\end{compactitem}
	\item [Aktivitet \arabic{aktivitetTeller}:] Design av nettstedet
	\begin{compactitem}
	\item Start: 29.01.2019
	\item Slutt: 12.02.2019
	\item Bemanning: Bjørnar
	\item Leveranse: Visuell profil, Designsystem, prototyping og Mockups. 
	\item Beskrivelse: med designguide.
	\addtocounter{aktivitetTeller}{1}
	\end{compactitem}
	\item [Aktivitet \arabic{aktivitetTeller}:] Database 
	\begin{compactitem}
	\item Start: 29.01.2019
	\item Slutt: 31.01.2019
	\item Bemanning: Hele gruppen
	\item Ansvarlig: Bereket
	\item Leveranse: Modell/skisser og ferdig database
	\item Beskrivelse: Planlegge og skissere databasen, deretter opprette planlagt database.
	\addtocounter{aktivitetTeller}{1}
	\end{compactitem}
	\item [Aktivitet \arabic{aktivitetTeller}:] Back-end
	\begin{compactitem}
	\item Start: 04.02.2019
	\item Slutt: 03.04.2019
	\item Bemanning: Bereket, Bjørnar
	\item Ansvarlig: Bereket
	\item Leveranse: xxx
	\item Beskrivelse: xxx
	\addtocounter{aktivitetTeller}{1}
	\end{compactitem}
	\item [Aktivitet \arabic{aktivitetTeller}:] Front-end
	\begin{compactitem}
	\item Start: 04.02.2019
	\item Slutt: 03.04.2019
	\item Bemanning: Line, Bjørnar
	\item Ansvarlig: Line
	\item Leveranse: sitemap
	\item Beskrivelse: SEO
	\addtocounter{aktivitetTeller}{1}
	\end{compactitem}
	\item [Aktivitet \arabic{aktivitetTeller}:] DevOps
	\begin{compactitem}
	\item Start: 25.03.2019
	\item Slutt: 08.04.2019
	\item Bemanning: Bjørnar
	\item Leveranse: Beta versjonen av nettstedet
	\item Beskrivelse: Laste opp og sette opp front-end og back-end på en server og peke domenet på denne serveren. Linux, Nginx, Mysql, PHP, phpmyadmin
	\addtocounter{aktivitetTeller}{1}
	\end{compactitem}
	\item [Aktivitet \arabic{aktivitetTeller}:] Brukertesting
	\begin{compactitem}
	\item Start: 09.04.2019
	\item Slutt: 15.04.2019
	\item Bemanning: Hele gruppen
	\item Ansvarlig: Line
	\item Leveranse: Dokumentasjon av brukertester
	\item Beskrivelse: Alle gruppedeltagere samt oppdragsgiver tester nettstedet og dets funksjonalitet.
	\addtocounter{aktivitetTeller}{1}
	\end{compactitem}
	\item [Aktivitet \arabic{aktivitetTeller}:] Ferdig produkt
	\begin{compactitem}
	\item Start: 15.04.2019
	\item Slutt: 29.04.2019
	\item Bemanning: Hele gruppen
	\item Ansvarlig: Bjørnar
	\item Leveranse: Nettstedet
	\item Beskrivelse: Ferdig produkt satt opp på Sirkus Media sitt domene
	\addtocounter{aktivitetTeller}{1}
	\end{compactitem}
	\item [Aktivitet \arabic{aktivitetTeller}:] Analyse av nytt nettsted
	\begin{compactitem}
	\item Start: 30.04.2019
	\item Slutt: 03.05.2019
	\item Bemanning: Bereket
	\item Leveranse: Rapport
	\item Beskrivelse: Dokumentasjon av analyse med forklaring av test-resultatene.
	\addtocounter{aktivitetTeller}{1}
	\end{compactitem}
	\item [Aktivitet \arabic{aktivitetTeller}:] Brukerveiledninger
	\begin{compactitem}
	\item Start: 30.04.2019
	\item Slutt: 03.05.2019
	\item Bemanning: Line
	\item Leveranse: 2 brukerveiledninger
	\item Beskrivelse: En brukerveiledning som beskriver hvordan man logger inn og endre innholdet. En brukerveiledning for hvordan man administrere nettstedet.
	\addtocounter{aktivitetTeller}{1}
	\end{compactitem}

\end{compactdesc}

\smallskip

Det kan være lurt å lage en prioritert plan, dvs. at noen av oppgavene vil bli bli gjennomført hvis det blir tid og anledning til det. Det er vanskelig å planlegge et prosjekt, nettopp derfor kan det være lurt at planen sier at dette vil vi oppnå som et minimum, og så kommer et antall prioriterte oppgaver. Dere vil også antagelig få behov for å re-planlegge underveis.  Det er også fornuftig å gjøre en kort risikoanalyse av prosjekt, og peke på kritiske faktorer og eventuelle flaskehalser. Vær spesielt oppmerksomme på at det er betenkelig å gjøre prosjektet avhenging av utstyr, programvare eller liknende som ikke er tilgjengelig ved prosjektstart.

\subsection*{Risikoanalyse}

Følgende kritiske faktorer og flaskehalser mener vi kan dukke opp i løpet av prosjektet:

Kortvarig sykdom: Ved eventuell kortvarig sykdom kan gruppedeltager forlenge fristen til en aktivitet med inntil 2 dager. 

Langvarig sykdom: Ved langvarig sykdom der gruppedeltager ikke leverer til avtalt tid, delegeres oppgaven videre til den deltageren som jobber med den aktiviteten som har lavest prioritet. Ved behov må prosjektplanen re-planlegges.

Intern konflikt: Ved en intern konflikt der deltagerne selv ikke kommer til enighet må gruppen inkludere veileder så raskt som mulig.

Dårlig kommunikasjon:  glknd

For korte frister: lanksf

\section*{Gjennomføring}

Her skal dere  fokusere på hvordan gruppa har tenkt seg gjennomføringen av prosjekt, ut over det som framgår av prosjektplanen. Dette kalles ofte for prosessen. Forholdet til arbeidsgiver står sentralt, hvordan skal dette foregå, hvor ofte skal man møtes, hva slags type tilbakemeldinger ønsker man seg etc.

Det er også viktig å redegjøre for hvordan man tenker seg rollene i selve gruppa, skal det f.eks. være en prosjektleder, skal det være en fast prosjektleder, eller skal rollene rullere? Tenker man å bruke en spesiell metodikk, for eksempel SCRUM, ved gjennomføringen av prosjektet og utviklingen av eventuell kode? Hva slags type infrastruktur er man blitt enige om, hvordan håndteres versjonskontroll og backup, hva med å jobbe distribuert?

En annen klassisk utfordring i et prosjektarbeid er hvordan man takler unntak. Hva gjør man hvis et gruppemedlem blir sykt i en kritisk periode, eller ikke leverer som avtalt? Tenker man seg muligheten av re-planlegging i visse situasjoner?

Til slutt kan det være greit å kommentere kort hvordan man tenker seg forholdet mellom veilederen og gruppa. 

--

Gruppen skal ha fortløpende kontakt med oppdragsgiver, minimum etter hver fullførte aktivitet. Da har oppdragsgiver mulighet til å gi tilbakemelding underveis, slik at det ikke vil komme store endringer nær innlevering. Dialogen med oppdragsgiver vil i hovedsak foregå over e-post, telefon og videosamtale. For slike type prosjekter er det ikke like stort behov for å ha møter fysisk, og gruppen vil ha like stort utbytte gjennom videosamtaler. Gruppen ønsker skriftlige tilbakemeldinger på design og funksjonalitet. 

Det er besluttet å ha en fast prosjektleder, da det er nødvendig å ha en fast person som følger opp prosessen og som har ansvar for dialogen med både veileder og oppdragsgiver. Gruppen har vedtatt at personen som er best egnet til dette er Bjørnar.

Deltagerne har i tillegg fått hver sitt overordnet ansvarsområdet når det kommer til gjennomføringen av nettstedet. Bereket har ansvar for back-end, Line for frond-end og Bjørnar for DevOps. Bjørnar vil også rullere med å jobbe med back-end og front-end, avhengig av behovet.

Alle medlemmene i gruppen skal delta med mest mulig likt antall timer for å gjennomføre prosjektet. Medlemmene skal møtes minst en  gang i uken for å jobbe sammen og oppgi status. Ved behov vil medlemmene møtes flere ganger i uken på studiestedet. I tillegg skal medlemmene ha dialog via Facebook Messenger og Google Hangouts.

Deltagerne har avtalt å ha møter med veileder hver andre uke, som et minimum. Ved behov vil gruppen og veileder møtes hyppigere. Dette avtales med veileder over e-post.

\end{document}
